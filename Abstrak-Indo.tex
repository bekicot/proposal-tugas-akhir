\chapter*{Abstrak}
Terumbu karang adalah ekosistem bawah laut yang terbentuk dari sekumpulan karang. Selain berfungsi sebagai ekosistem di bawah laut, terumbu karang juga berfungsi sebagai pemecah gelombang. Selama ini metode yang ada untuk memprediksi tinggi gelombang \emph{runup} pada terumbu karang dengan menggunakan Pembelajaran Mesin masih tergolong baru. Metode yang saat ini digunakan dibagi menjadi 2. Metode pertama dilakukan dengan pendekatan klasik, yakni dengan melakukan eksperimen dan observasi, lalu dicari model matematisnya. Sedangkan model yang kedua dilakukan dengan pendekatan \emph{soft computing}. Pada Tugas Akhir ini akan menggunakan Pembelajaran Mesin untuk memprediksi \emph{runup} gelombang di atas terumbu karang, dengan metode yang dipakai adalah \emph{Artificial Neural Network}. Data yang digunakan adalah data dari hasil eksperimen yang dilakukan oleh Demirbilek (2007) \cite{DemirbilekReport} yang dilakukan pada laboratorium dinamika untuk meneliti redaman gelombang oleh terumbu karang.
\vspace{0.5 cm}
\begin{flushleft}
{\textbf{Kata Kunci:} Terumbu karang, Gelombang Air Laut, Neural Network.}
\end{flushleft}