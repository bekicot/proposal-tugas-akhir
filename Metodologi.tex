\chapter{Metodologi dan Desain Sistem}
Pada TA ini akan di 
\section{Deskripsi Data}
Data yang akan digunakan dalam aplikasi neural network di TA ini adalah data hasil analisis dari eksperimen yang di lakukan oleh US Army Corps of Engineer pada Agustus - September 2006. Analisa dilakukan oleh Demirbilek et al. dan di tulis dalam laporan yang berjudul "Laboratory Study of Wind Effect on Runup over Fringing Reefs".

\subsection{Kondisi eksperimen}
9 sensor dipasang disepanjang tabung eksperimen dengan jarak yang bervariasi. Di bagian atas gelombang dipasang blower angin yang disertai sensor kecepatan angin. Eksperimen dibagi menjadi 3 bagian. Eksperimen pertama dilakukan hanya menggunakan variabel gelombang, dengan kecepatan angin 0. Eksperimen kedua, dilakukan hanya menggunakan variabel angin. Selanjutnya eksperimen ketiga adalah gabungan dari perubahan variable gelombang dan variabel angin.

\subsection{Hasil Analisa Data}
    \begin{table}
    \caption{Sampel data hasil analisa.}
    \begin{center}
      \begin{tabular}{|l|l|l|l|l|l|l|l|}
      \hline
      TestId & H & T & WL & $R_{max}$ & Wind \\ \hline
      Test99 & 5.5 & 1.25 & 50.0 & 1.5 & 7.1 \\ \hline
      Test100 & 6.0 & 1.0 & 50.0 & 1.6 & 6.9 \\ \hline
      Test101 & 3.3 & 1.0 & 50.0 & 0.7 & 5.3 \\ \hline
      Test102 & 8.2 & 2.5 & 53.1 & 8.4 & 7.0 \\ \hline
      Test103 & 8.6 & 2.0 & 53.1 & 7.8 & 7.1 \\ \hline
      Test104 & 8.0 & 1.75 & 53.1 & 6.7 & 5.8 \\ \hline
      Test105 & 7.8 & 1.5 & 53.1 & 5.9 & 6.3 \\ \hline
      Test106 & 5.4 & 1.5 & 53.1 & 4.0 & 6.7 \\ \hline
      Test107 & 6.3 & 1.25 & 53.1 & 4.5 & 6.8 \\ \hline
      Test108 & 6.6 & 1.0 & 53.1 & 6.4 & 6.7 \\ \hline
      Test109 & 3.8 & 1.0 & 53.1 & 5.3 & 6.5 \\ \hline
      \end{tabular}
      \end{center}
    \end{table}
  \FloatBarrier
  Dari laporan "Laboratory Study of Wind Effect on Runup over Fringing Reefs" \cite{DemirbilekReport} data yang di hasilkan berupa data hasil analisa yang berasal dari raw data yang merupakan time series. Pada table tersebut $H$ merupakan tinggi gelombang, $T$ merupakan spectral peak periods, WL merupakan Wave Length, dan Rmax adalah ketinggian maximum dari runup. $H$, $WL$, dan $R_{max}$ merupakan dalam $cm$.

\section{Flowchart sistem}
\begin{figure}
  \caption{Flowchart Sistem}
  \begin{center}
    \tikzstyle{decision} = [diamond, draw, fill=blue!20, 
        text width=4.5em, text badly centered, node distance=3cm, inner sep=0pt]
    \tikzstyle{block} = [rectangle, draw, fill=blue!20, 
        text width=10em, text centered, rounded corners, minimum height=6em]
    \tikzstyle{blockTraining} = [rectangle, draw, fill=red!20, 
        text width=10em, text centered, rounded corners, minimum height=6em]
    \tikzstyle{line} = [draw, -latex']
    \tikzstyle{cloud} = [draw, ellipse,fill=red!20, node distance=3cm,
        minimum height=2em]
        
    \begin{tikzpicture}[node distance = 3cm, auto]
        % Place nodes
        \node [block] (init) {Mulai};
        \node [block, below of=init] (bacaData) {Membaca data report dari csv file};
        \node [block, below of=init] (transformasiData) {Transformasi data menjadi matrix};
        \node [block, below of=transformasiData] (splitTrainingTesting) {Split data 80 persen training, 20 persen testing};
        \node [block, below of=splitTrainingTesting] (setupEpochLr) {Menentukan jumlah epoch dan learning rate};
        \node [blockTraining, below of=setupEpochLr] (training) {Training Data};
        \node [block, below of=training] (testing) {Testing};
        \node [block, below of=testing] (stop) {Stop};
      
        % Draw edges
        \path [line] (init) -- (bacaData);
        \path [line] (bacaData) -- (transformasiData);
        \path [line] (transformasiData) -- (splitTrainingTesting);
        \path [line] (splitTrainingTesting) -- (setupEpochLr);
        \path [line] (setupEpochLr) -- (training);
        \path [line] (training) -- (testing);
        \path [line] (testing) -- (stop);
    \end{tikzpicture}
  \end{center}
\end{figure}
\begin{figure}
  \caption{Flowchart Training}
  \begin{center}
    \tikzstyle{block} = [rectangle, draw, fill=blue!20, 
        text width=10em, text centered, rounded corners, minimum height=6em]
    \tikzstyle{blockTraining} = [rectangle, draw, fill=red!20, 
        text width=14em, text centered, rounded corners, minimum height=2em]
    \tikzstyle{stopTrainng} = [rectangle, draw, fill=red!20, 
        text width=3em, text centered, rounded corners,  distance=10cm]
    \tikzstyle{line} = [draw, -latex']
    \tikzstyle{cloud} = [draw, ellipse,fill=red!20, node distance=2.3cm,
        minimum height=2em]
    \tikzstyle{decision} = [diamond, draw, fill=blue!20, 
        text width=4em, text badly centered, node distance=2.5cm, inner sep=0pt]
        
    \begin{tikzpicture}[node distance = 2cm, auto]
        % Place nodes
        \node [blockTraining] (init) {Mulai};
        \node [blockTraining, below of=init] (randomWeight) {Inisialisasi $weight$ dengan random data};
        \node [blockTraining, below of=randomWeight] (IW){Mengalikan Input Layer Dengan $Weight$ hidden layer};
        \node [blockTraining, below of=IW] (aktivasi){Jalankan fungsi aktivasi hidden layer};
        \node [blockTraining, below of=aktivasi] (HW){Mengalikan Input Layer Dengan $Weight$ untuk $output$};
        \node [blockTraining, below of=HW] (aktivasiOutput){Mengalikan Input Layer Dengan $Weight$ untuk $output$};
        \node [blockTraining, below of=aktivasiOutput] (cost){Kalkulasi cost};
        \node [blockTraining, below of=cost] (stop){Stop};
  
        % Draw edges
        \path [line] (init) -- (randomWeight);
        \path [line] (randomWeight) -- (IW);
        \path [line] (IW) -- (aktivasi);
        \path [line] (aktivasi) -- (HW);
        \path [line] (HW) -- (aktivasiOutput);
        \path [line] (aktivasiOutput) -- (cost);
        \path [line] (cost) -- (stop);
    \end{tikzpicture}
  \end{center}
\end{figure}
\FloatBarrier
