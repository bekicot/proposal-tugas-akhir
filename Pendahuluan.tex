\chapter{Pendahuluan}
\section{Latar Belakang}

Terumbu karang adalah ekosistem bawah laut yang terbentuk dari sekumpulan karang. Selain berfungsi sebagai ekosistem di bawah laut, terumbu karang juga berfungsi sebagai pemecah gelombang. Sebagian besar kepulauan di wilayah pasific di kelilingi oleh terumbu karang yang tumbuh di laut dangkal yang dekat dengan pantai \cite{DemirbilekBoussinesq}. 

% $_GAMBAR_LANDSAT_$

Selama ini metode yang ada untuk memprediksi tinggi gelombang runup pada terumbu karang masih tergolong baru. Metodenya sendiri terbagi menjadi 2. Metode yang pertama dilakukan dengan pendekatan klasik, dan di lakukan secara analitis. Yakni dengan melakukan eksperimen dan observasi, lalu di cari model matematika yang tepat. Model yang demikian sulit untuk dikembangkan dan beradaptasi dengan kondisi lingkungan yang berbeda. Prediksi yang didapat dari model yang demikian pun masih belum sempurna \cite{DemirbilekBoussinesq}. Dalam \ref{DemirbilekBoussinesq}. Sedangkan metode yang kedua dilakukan dengan pendekatan soft computing.

Pada Tugas Akhir ini akan menggunakan Pembelajaran Mesin untuk memprediksi runup gelombang di atas terumbu karang, dengan metode yang dipakai adalah Artificial Neural Network. Hasilnya akan dianalisa dan dibandingkan dengan tinggi runup sebenarnya, sesuai dengan hasil observasi. Setelah itu, dari hasil prediksi, akan dilihat pula efisiensi dari terumbu karang dalam meredam gelombang.

\section{Perumusan Masalah}
Rumusan masalah yang ingin saya angkat adalah
\begin{enumerate}
    \item Seberapa efisien terumbu karang dalam meredam gelombang?
\end{enumerate}
\section{Tujuan}
Berikut adalah tujuan yang ingin dicapai pada penulisan proposal/TA.
\begin{enumerate}
    \item Mengetahui akurasi Artificial Neural Network untuk prediksi runup gelombang pada terumbu.
    \item Mengetahui efisiensi dari terumbu karang dalam meredam gelombang.
\end{enumerate}
\section{Batasan Masalah}
Untuk memastikan hasil yang cukup akurat
\begin{enumerate}
    \item Data berasal dari experiment yang di publikasikan oleh badan pertahanan Amerika Serikat
    \item Data experiment berada dalam kondisi simulasi keadaan terumbu karang di guam dengan rasio 1:64.
\end{enumerate}
\section{Rencana Kegiatan}
Rencana kegiatan yang akan saya lakukan adalah sebagai berikut:
\begin{itemize}
    \item Studi literatur: Pengumpulan informasi dan referensi.
    \item Penentuan Topik
    \item Analisis dan Perancangan Sistem
    \item Implementasi Sistem
    \item Analisa Hasil Implementasi
    \item Penulisan Laporan
\end{itemize}
\section{Jadwal Kegiatan}

Laporan proposal ini akan dijadwalkan sesuai dengan tabel berikut:

 
\begin{table}[h!]
  \centering
    \caption{Jadwal kegiatan proposal tugas akhir}
  \label{Novella}
  \begin{tabular}{|c|m{2.5cm}|m{0.01cm}|m{0.01cm}|m{0.01cm}|m{0.01cm}|m{0.01cm}|m{0.01cm}|m{0.01cm}|m{0.01cm}|m{0.01cm}|m{0.01cm}|m{0.01cm}|m{0.01cm}|m{0.01cm}|m{0.01cm}|m{0.01cm}|m{0.01cm}|m{0.01cm}|m{0.01cm}|m{0.01cm}|m{0.01cm}|m{0.01cm}|m{0.01cm}|m{0.01cm}|m{0.01cm}|}
    \hline
    \multirow{2}{*}{\textbf{No}} & \multirow{2}{*}{\textbf{Kegiatan}} & \multicolumn{24}{|c|}{\textbf{Bulan ke-}} \\
    \hhline{~~------------------------}
    {} & {} & \multicolumn{4}{|c|}{\textbf{1}} & \multicolumn{4}{|c|}{\textbf{2}} & \multicolumn{4}{|c|}{\textbf{3}} & \multicolumn{4}{|c|}{\textbf{4}} & \multicolumn{4}{|c|}{\textbf{5}} & \multicolumn{4}{|c|}{\textbf{6}}\\
    \hline
    1 & Studi Literatur & \cellcolor{blue!25} & \cellcolor{blue!25} & \cellcolor{blue!25} & \cellcolor{blue!25}& \cellcolor{blue!25} & \cellcolor{blue!25} & \cellcolor{blue!25} & \cellcolor{blue!25}& \cellcolor{blue!25} & \cellcolor{blue!25} & \cellcolor{blue!25} & \cellcolor{blue!25}& \cellcolor{blue!25} & \cellcolor{blue!25} & \cellcolor{blue!25} & \cellcolor{blue!25}& \cellcolor{blue!25} & \cellcolor{blue!25} & \cellcolor{blue!25} & \cellcolor{blue!25}& \cellcolor{blue!25} & \cellcolor{blue!25} & \cellcolor{blue!25} & \cellcolor{blue!25}\\
    \hline
    2 & Analisis dan Perancangan Sistem &  {} & {} & {} & {}  & \cellcolor{blue!25} & \cellcolor{blue!25} & \cellcolor{blue!25} & \cellcolor{blue!25} & \cellcolor{blue!25} & \cellcolor{blue!25} & \cellcolor{blue!25} & \cellcolor{blue!25} & {} & {} & {} & {}& {} & {} & {} & {}& {} & {} & {} & {}\\
    \hline
    3 & Implementasi Sistem &  {} & {} & {} & {} & {} & {} & {} & {}& \cellcolor{blue!25} & \cellcolor{blue!25} & \cellcolor{blue!25} & \cellcolor{blue!25} & \cellcolor{blue!25} & \cellcolor{blue!25} & \cellcolor{blue!25} & \cellcolor{blue!25} & {} & {} & {} & {}& {} & {} & {} & {}\\
    \hline
    4 & Analisa Hasil Implementasi &  {} & {} & {} & {} & {} & {} & {} & {}& {} & {} & {} & {} & \cellcolor{blue!25} & \cellcolor{blue!25} & \cellcolor{blue!25} & \cellcolor{blue!25} & \cellcolor{blue!25} & \cellcolor{blue!25} & \cellcolor{blue!25} & \cellcolor{blue!25} & {} & {} & {} & {}\\
    \hline
    5 & Penulisan Laporan & {} & {} & {} & {} & \cellcolor{blue!25} & \cellcolor{blue!25} & \cellcolor{blue!25} & \cellcolor{blue!25}& \cellcolor{blue!25} & \cellcolor{blue!25} & \cellcolor{blue!25} & \cellcolor{blue!25}& \cellcolor{blue!25} & \cellcolor{blue!25} & \cellcolor{blue!25} & \cellcolor{blue!25}& \cellcolor{blue!25} & \cellcolor{blue!25} & \cellcolor{blue!25} & \cellcolor{blue!25}& \cellcolor{blue!25} & \cellcolor{blue!25} & \cellcolor{blue!25} & \cellcolor{blue!25}\\
    \hline
  \end{tabular}
\end{table}

